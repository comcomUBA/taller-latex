\section{Introducción}\label{sec:Introducción}

\begin{frame}
    \frametitle{Introducción}
        \begin{tcolorbox}[colframe=color1]
            \begin{center}
                ¿Qué \textit{no} es \LaTeX ?
            \end{center}
        \end{tcolorbox}
        
        \pause
        \begin{itemize}
            \item Un procesador de documentos o texto.
            \item Un lenguaje de programación.
            \item Algo que no se entiende muy bien como funciona ni para que sirve. 
            % malísimo el último ítem pero me pone nerviosa si no son 3 jaja
        \end{itemize}
        
    \end{frame}

\begin{frame}
\frametitle{Introducción}
    \begin{tcolorbox}[colframe=color1]
        \begin{center}
            ¿Qué es \LaTeX ?
        \end{center}
    \end{tcolorbox}
    
    \begin{itemize}
        \item Es una herramienta o sistema de preparación de documentos.
        \item Se usa en lo común en ambientes académicos o profesionales.
        \item La idea es escribir código que después se compila y produce nuestro documento.
        %\item WYSIWYG ... ejemplos, editor de texto plano (.txt), planilla de cálculo, etc
        %\item WYWIWYG ... (Quizás contar que es un lenguaje de propósito específico, y también mencionar Markdown y Typst)
        \end{itemize}    

    \begin{figure}[h]
        \centering
        \includegraphics[width=0.4\textheight]{../images/lion.png}
        \caption{Dibujo del león mascota de CTAN por Duane Bibby.}
    \end{figure}

\end{frame}

\begin{frame}
    \frametitle{Introducción}
    \begin{tcolorbox}[colframe=color1]
        \begin{center}
            ¿Por qué rayos usaría \LaTeX ?
        \end{center}
    \end{tcolorbox}

    \begin{itemize}
        \item Permite formatear fórmulas matemáticas complejas con facilidad.
        \item Permite crear referencias y matenerlas coherentes a través del texto.
        \item Extremadamente customizable y orientado a la comunidad.
        \item Es gratis (y no se roba tus datos para entrenar IAs).
    \end{itemize}
\end{frame}

\begin{frame}
\frametitle{Introducción}
    Ejemplos de dónde se usa o puede usar \LaTeX ...
    \begin{itemize}
    \item Escribir documentación
    \item Papers
    \item Los exámenes de las materias
    \item Las diapos de las clases
    \item ...
    \end{itemize}
    Indicar para qué cosas lo usamos en la carrera, o dónde podríamos querer usarlo.
    \begin{itemize}
    \item Escribir apuntes
    \item Armar informes
    \item Diapos para presentaciones (como los concursos docentes)
    \item Armar un CV
    \item ...
    \end{itemize}
\end{frame}
