\section{Introducción}\label{sec:Introducción}

\begin{frame}
    \frametitle{Introducción}
        \begin{tcolorbox}[colframe=color1]
            \begin{center}
                ¿Qué \textit{no} es \LaTeX ?
            \end{center}
        \end{tcolorbox}
        \begin{itemize}
            \item Un procesador de documentos o texto.
            \item Un lenguaje de programación (solamente).
            \item Algo que no se entiende para que sirve.
        \end{itemize}
        
    \end{frame}

\begin{frame}
\frametitle{Introducción}
    \begin{tcolorbox}[colframe=color1]
        \begin{center}
            ¿Qué es \LaTeX ?
        \end{center}
    \end{tcolorbox}
    
    \begin{itemize}
        \item Es una herramienta o sistema de preparación de documentos.
        \item Se usa comunmente en ambientes académicos o profesionales.
        \item La idea es escribir código que después se compila, y produce nuestro documento.
        \end{itemize}    

    \begin{figure}[h]
        \centering
        \includegraphics[width=0.4\textheight]{../images/lion.png}
        \caption{Dibujo del león mascota de CTAN por Duane Bibby.}
    \end{figure}

\end{frame}

\begin{frame}
    \frametitle{Introducción}
    \begin{tcolorbox}[colframe=color1]
        \begin{center}
            ¿Por qué rayos usaría \LaTeX ?
        \end{center}
    \end{tcolorbox}

    \begin{itemize}
        \item Permite formatear fórmulas matemáticas complejas con facilidad.
        \item Permite crear referencias y matenerlas coherentes a través del texto.
        \item Extremadamente customizable y orientado a la comunidad.
        \item Es gratis (y no se roba tus datos para entrenar IAs).
    \end{itemize}
\end{frame}

\begin{frame}
    \frametitle{Introducción}
    \framesubtitle{Y lo mas importante\dots}
    \begin{figure}[h]
        \centering
        \includegraphics[width=\textwidth]{../images/submission.png}
        \caption{En el proceso de presentación de un paper, muchas veces es obligatorio entregarlo con un formato específico. Para asegurarlo, se provee un template de \LaTeX.}
    \end{figure}
\end{frame}

\begin{frame}
\frametitle{Introducción}
    En la facultad, se usa \LaTeX\ para:
    \begin{itemize}
    \item Los exámenes de las materias.
    \item Las diapos de las clases.
    \item Armar informes y TPs.
    \item Escribir apuntes.
    \item ...
    \end{itemize}
\end{frame}
