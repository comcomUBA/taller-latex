\section{Introducción}\label{sec:Introducción}

\begin{frame}
\frametitle{Introducción}
    \begin{tcolorbox}[colframe=color1]
        \begin{center}
            ¿Qué es \LaTeX ?
        \end{center}
    \end{tcolorbox}
    
    Luego de una breve intro se puede comentar qué es WYSIWYG y WYWIWYG
    
    \begin{itemize}
    \item WYSIWYG ... ejemplos, editor de texto plano (.txt), planilla de cálculo, etc
    \item WYWIWYG ... (Quizás contar que es un lenguaje de propósito específico, y también mencionar Markdown y Typst)
    \end{itemize}    
\end{frame}

\begin{frame}
\frametitle{Introducción}
    Ejemplos de dónde se usa o puede usar \LaTeX ...
    \begin{itemize}
    \item Escribir documentación
    \item Papers
    \item Los exámenes de las materias
    \item Las diapos de las clases
    \item ...
    \end{itemize}
    Indicar para qué cosas lo usamos en la carrera, o dónde podríamos querer usarlo.
    \begin{itemize}
    \item Escribir apuntes
    \item Armar informes
    \item Diapos para presentaciones (como los concursos docentes)
    \item Armar un CV
    \item ...
    \end{itemize}
\end{frame}
