\section{Ejercicios}\label{ejercicios}

\begin{frame}
\frametitle{Ejercicio}
\framesubtitle{Pseudocódigo}

Por último, queremos poder tomar un paso atrás y escribir en abstracto un pesudocógido, más allá del lenguaje en el que lo implementemos.

\

Para ello, podemos utilizar el paquete \texttt{algpseudocode} con el entorno \texttt{algorithmic}. Y dentro del mismo, incorporamos el pseudocódigo de nuestro programa.

\begin{multicols}{2}

Por ejemplo, si queremos escribir un condicional $if$-$else$, usamos \\

\textbackslash If\{a\} \\
$~~~~~~$\textbackslash State b \\
\textbackslash Else \\
$~~~~~~$\textbackslash State c \\
\textbackslash EndIf \\

\pause

O si queremos un ciclo de iteración $for$ \\

\textbackslash For\{$i \in 1,\ldots,n$\} \\
$~~~~~~$\textbackslash State $a \gets a + i$ \\
\textbackslash EndFor

\

\end{multicols}
\end{frame}

\begin{frame}
\frametitle{Ejercicio}
\framesubtitle{Pseudocódigo}

Con ello, escribamos el siguiente pseudocódigo:

\begin{tcolorbox}[colframe=color1]
\begin{center}
\begin{algorithmic}
    \Function{sumarTuplas}{tuplas}
        \State $suma \gets (0,0)$
        \If{$tupla \neq \{\}$}
        \For{$i = 1, \ldots, |tuplas|$}
            \State $suma_1 \gets tupla[i]_1$
            \State $suma_2 \gets tupla[i]_2$
        \EndFor
        \EndIf
        \State \Return $suma$
    \EndFunction
\end{algorithmic}
\end{center}
\end{tcolorbox}

\end{frame}