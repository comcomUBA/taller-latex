\section{Ejercicios}\label{ejercicios}

\begin{frame}
\frametitle{Ejercicio}
\begin{center}

Proponer algunos ejercicios. Ideas:

\begin{itemize}
%\item Algo similar a un hola mundo
%\item Un documento simple A4 con título, autoría y fecha.
\item Un ejemplo de fórmula matemática "medio compleja" (hacer cosas subíndices y supraíndices), espaciado. Cómo usar el modo mátemático
\item Ejemplo donde se use colores/formatos + armar macros con ellos
\item Usar matrices o tablas
%\item Armar un algoritmo con un paquete para hacer pseudocógido
\item Incluir una imagen, varias imágenes. cambiar su tamaño o posición.
%\item Incluir código, como configurar el diseño con el paquete lstlisting
\item Armar diapos (charlar sobre distintos tipos de documentos, beamer, article, minimal, etc)
\item Usar Tikz (algo básico)
\end{itemize}

\end{center}
\end{frame}

\begin{frame}
\frametitle{Ejercicio - 1}
\framesubtitle{Modelo de TP}

Vamos a hacer un simulacro de informe. El TP que tenemos que entregar gira en torno a algoritmos vistos en una materia, y en su informe tenemos que escribir el pseudocódigo de uno de estos algoritmos, junto con su implementación. \pause

\

Queremos tener un informe para entregar que conste de:

\begin{itemize}
\item Carátula con nombre del TP, integrantes, fecha. \pause
\item Una primer página con una síntesis del contenido del informe, con palabras clave del mismo y una tabla de contenidos. \pause
\item A continuación, tres secciones. La primera con la consigna, la segunda con el pseudocódigo y la tercera con la implementación.
\end{itemize}

\end{frame}