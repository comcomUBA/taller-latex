\section{Ejercicios}\label{ejercicios}

\begin{frame}
\frametitle{Ejercicio}
\framesubtitle{Pseudocódigo}

Por último, queremos poder tomar un paso atrás y escribir en abstracto un pesudocódigo, más allá del lenguaje en el que lo implementemos.

\

Para ello, podemos utilizar el \textcolor{codeorange}{paquete} \texttt{algpseudocode} con el \textcolor{codeorange}{entorno} \texttt{algorithmic}. Y dentro del mismo, incorporamos el pseudocódigo de nuestro programa (noten el uso de mayúsculas para cada comando).

\pause

\


\colorbox{color2}{Un bloque $function$:} \\

\textbackslash begin\{algorithmic\} \\
\textbackslash Function\{\texttt{nombre}\}\{$p_1, p_2,\ldots$\} \\
\hspace*{15pt} \textbackslash State $suma \gets 2$ \\
\textbackslash EndFunction \\
\textbackslash end\{algorithmic\}

\end{frame}

\begin{frame}
\frametitle{Ejercicio}
\framesubtitle{Pseudocódigo}


\begin{multicols}{2}

\colorbox{color2}{Un condicional $if$-$else$:} \\

\textbackslash If\{a\} \\
\hspace*{15pt} \textbackslash State b \\
\textbackslash Else \\
\hspace*{15pt} \textbackslash State c \\
\textbackslash EndIf \\

\

\

\colorbox{color2}{Ciclo $for$:} \\

\textbackslash For\{$i \in 1,\ldots,n$\} \\
\hspace*{15pt} \textbackslash State $a \gets a + i$ \\
\textbackslash EndFor

\

\colorbox{color2}{Retorno:} \\

\textbackslash State \textbackslash Return $res$ \\


\end{multicols}
\end{frame}

\begin{frame}
\frametitle{Ejercicio}
\framesubtitle{Pseudocódigo}

Con ello, escribamos el siguiente pseudocódigo:

\begin{tcolorbox}[colframe=color1]
\begin{center}
\begin{algorithmic}
    \Function{sumarTuplas}{tuplas}
        \State $suma \gets (0,0)$
        \If{$tupla \neq \{\}$}
        \For{$i = 1, \ldots, |tuplas|$}
            \State $suma_1 \gets tupla[i]_1$
            \State $suma_2 \gets tupla[i]_2$
        \EndFor
        \EndIf
        \State \Return $suma$
    \EndFunction
\end{algorithmic}
\end{center}
\end{tcolorbox}

Fijense que algunos símbolos necesitan ser escapados o escribirse en modo matemático

\end{frame}